\documentclass[letterpaper]{article}
\usepackage{amsmath}
\usepackage{tikz}
\usepackage{epigraph}
\usepackage{lipsum}
\usepackage{hyperref}
\usepackage{tocloft}

\usepackage{setspace, amsmath}

\usepackage[centering,includeheadfoot,margin=2cm]{geometry}
\usepackage{xcolor}
\usepackage{calc,blindtext}

\renewcommand\epigraphflush{flushright}
\renewcommand\epigraphsize{\normalsize}
\setlength\epigraphwidth{0.6\textwidth}

\definecolor{titlepagecolor}{cmyk}{1,.60,0,.40}

\DeclareFixedFont{\titlefont}{T1}{ppl}{b}{it}{1.0in}

\makeatletter
\def\printauthor{%
    {\large \@author}}
\makeatother
\author{%
    Nico Taljaard \\
    10153285 \vspace{20pt} \\
    Gerhard Smit \\
    12282945 \vspace{20pt} \\
    Martin Schoeman \\
    10651994 \\
}

% The following code is borrowed from: http://tex.stackexchange.com/a/86310/10898

\newcommand\titlepagedecoration{%
	\begin{tikzpicture}[remember picture,overlay,shorten >= -10pt]
	
		\coordinate (aux1) at ([yshift=-15pt]current page.north east);
		\coordinate (aux2) at ([yshift=-410pt]current page.north east);
		\coordinate (aux3) at ([xshift=-4.5cm]current page.north east);
		\coordinate (aux4) at ([yshift=-150pt]current page.north east);
		
		\begin{scope}[titlepagecolor!40,line width=12pt,rounded corners=12pt]
			\draw
			  (aux1) -- coordinate (a)
			  ++(225:5) --
			  ++(-45:5.1) coordinate (b);
			\draw[shorten <= -10pt]
			  (aux3) --
			  (a) --
			  (aux1);
			\draw[opacity=0.6,titlepagecolor,shorten <= -10pt]
			  (b) --
			  ++(225:2.2) --
			  ++(-45:2.2);
		\end{scope}
			\draw[titlepagecolor,line width=8pt,rounded corners=8pt,shorten <= -10pt]
			  (aux4) --
			  ++(225:0.8) --
			  ++(-45:0.8);
		\begin{scope}[titlepagecolor!70,line width=6pt,rounded corners=8pt]
			\draw[shorten <= -10pt]
			  (aux2) --
			  ++(225:3) coordinate[pos=0.45] (c) --
			  ++(-45:3.1);
			\draw
			  (aux2) --
			  (c) --
			  ++(135:2.5) --
			  ++(45:2.5) --
			  ++(-45:2.5) coordinate[pos=0.3] (d);   
			\draw 
			  (d) -- +(45:1);
		\end{scope}
	\end{tikzpicture}
}

\begin{document}

\begin{titlepage}

\noindent
\titlefont Laminin \par
\epigraph{XGame - Derivco \\ Corpse Slasher \\ Architecture Requirements Specification.}%
{\textit{ 01/08/2014 }\\ \textsc{ }}
\null\vfill
\vspace*{4cm}
\noindent
\hfill
\begin{minipage}{0.35\linewidth}
    \begin{flushright}
        \printauthor
    \end{flushright}
\end{minipage}
%
\begin{minipage}{0.02\linewidth}
    \rule{1pt}{125pt}
\end{minipage}
\titlepagedecoration
\end{titlepage}

% % % % % % % % % % % % % % %
% 							%
%	Remainder of document	%
% 							%
% % % % % % % % % % % % % % % 

	\newpage
		{\LARGE \bf Change Log}\\[2em]
		
		\begin{tabbing}
			\hspace*{2.5cm}\=\hspace*{2.5cm}\=\hspace*{8cm}\=\hspace*{3cm} \kill
			17/05/2014	\> Version 1.0	\> Document Created 							\> Nico Taljaard \\
			17/05/2014	\> Version 1.0	\> Access channel \& Quality requirements		\> Nico Taljaard \\
			18/05/2014	\> Version 1.0	\> Integration requirements, Architectural constraints \\
						\>				\>	\& added quality requirements 				\> Nico Taljaard \\
		\end{tabbing}
		
	\newpage
		\renewcommand\contentsname{TABLE OF CONTENTS}
		\newcommand\contentsnameLC{\colorbox{blue}{\makebox[\textwidth-2\fboxsep][l]{\bfseries\color{white} Table of Contents}}}
		
		\renewcommand{\cftdot}{}
		\hypersetup{linktocpage}
		\tableofcontents
		
		\begin{flushleft}
			\LARGE\href{https://github.com/njTaljaard/Laminin_CorpseSlasher/}{Git repository: Laminin - Corpse Slasher}
		\end{flushleft}
		
	\newpage
		
		\section*{\colorbox{blue}{\makebox[\textwidth-2\fboxsep][l]{\bfseries\color{white} Access channel requirements }}} \addcontentsline{toc}{section}{Access channel requirements}
		\vspace{0.1in}
		
		The system will be accessible by human users through the following channels:
		\begin{itemize}
			\item From a desktop application running on any Windows/Linux operating system.
			\item From a mobile device running Android application clients.
		\end{itemize}
		
		\vspace{0.2in}
		\section*{\colorbox{blue}{\makebox[\textwidth-2\fboxsep][l]{\bfseries\color{white} Quality requirements}}} \addcontentsline{toc}{section}{Quality requirements}
		\vspace{0.1in}
		
			The quality requirements are the requirements around the quality attributes of the systems and the
			services it provides.
			
			\subsection*{Performance}
			\addcontentsline{toc}{subsection}{Performance}
			\vspace{0.1in}
			
			\begin{enumerate}
				\item Desktop \& Mobile application:
					\begin{itemize}
						\item Should not go under a playable frames rate of 60.
						\item Should not have high data transfers for sending \& receiving information.
						\item Should not require more the 4 logical processing cores.
					\end{itemize}
			
				\item Desktop application:
					\begin{itemize}
						\item Should not use more the 3GB of system memory.
					\end{itemize}
			
				\item Mobile application:
					\begin{itemize}
						\item Should not use more the 2GB of system memory.
						\item Should not use access battery usage.
					\end{itemize}
					
				\item Backend server:
					\begin{itemize}
						\item Should be able to handle at least 100 connections.
						\item Should not use more then 10GB of system memory to handle connections.
						\item Database updates should be complete within a 1 second time frame.
						\item Database requests should be complete within 5 second time frame.
					\end{itemize}
			\end{enumerate}
			
			\subsection*{Reliability}
			\addcontentsline{toc}{subsection}{Reliability}
			\vspace{0.1in}
			
			\begin{enumerate}
				\item Desktop \& Mobile:
					\begin{itemize}
						\item Client should be able to recover from a crash within 30 seconds.
						\item Retry opening connection within 5 seconds if connection loss.
						\item Reload assets if data is lost.
					\end{itemize}
				\item Backend Server:
					\begin{itemize}
						\item Should provide 99\% connection availability.
						\item Should provide 99\% OAuth login availability.
						\item Should not crash due to over welling client connections.
						\item Should not crash due to undefined incoming data.
					\end{itemize}
			\end{enumerate}
			
			\subsection*{Scalability}
			\addcontentsline{toc}{subsection}{Scalability}
			\vspace{0.1in}
			
			\begin{enumerate}
				\item The server should be able to add multiple servers to handle clients.
				\item The server should scale to any amount of client connections.
			\end{enumerate}
			
			\subsection*{Security}
			\addcontentsline{toc}{subsection}{Security}
			\vspace{0.1in}
			
			General security conditions.
			\begin{enumerate}
				\item Use OAuth LDAP for connection through social media e.g. Facebook, Twitter, Google+.
				\item Clean up all incoming data on client and server for incorrect formats, data types \& lengths.
				\item Test if the incoming data comes from authenticated user.
				\item Local authentication users needs secure connection for login and password retrieval.
			\end{enumerate}
			
			\subsection*{Maintainability}
			\addcontentsline{toc}{subsection}{Maintainability}
			\vspace{0.1in}
			
			\begin{enumerate}
				\item Database off site backups should be creatable.
				\item Desktop \& mobile application should be update able with further maps.
			\end{enumerate}
			
			\vspace{0.2in}
			\section*{\colorbox{blue}{\makebox[\textwidth-2\fboxsep][l]{\bfseries\color{white} Integration requirements}}} \addcontentsline{toc}{section}{Integration requirements}
			\vspace{0.1in}
			
			\subsection*{Integration channels}
			\addcontentsline{toc}{subsection}{Integration channels}
			\vspace{0.1in}
			
			This system will be able to access:
			\begin{enumerate}
				\item A local MySQL datbase to access player information \& leader board.
				\item External social media LDAP servers for players who would like to share their achievements.
				\item External application connection to the server should only be encrypted during login.
			\end{enumerate}
		
			\vspace{0.2in}
			\subsection*{Protocols}
			\addcontentsline{toc}{subsection}{Protocols}
			\vspace{0.1in}
		
			This system will use the following protocols:
			\begin{enumerate}
				\item Java client server for communication between the Desktop \& Mobile application and the backend server. 
				\item JSON is used for the data encoding 
				\item LDAP protocol is used by the LDAP adapter to communicate with the LDAP database.
				\item SQL to communicate with the relational datebase.
			\end{enumerate}
		
			\vspace{0.2in}
			\subsection*{API specification}
			\addcontentsline{toc}{subsection}{API specification}
			\vspace{0.1in}
						\begin{enumerate}
			\item \textbf{Client Connection}
			\\This will create a connection object that will be sent to the listening server checking the validity of the connection such as username and password, or if they chose one of the other options such as logging in through social media checking if that user has an account there, if successful returns success and allows the user to continue. The user can also choose to create an account and then the client sends a request object and allows the user to fill in all appropriate details and an account is made. If the user forgot his password they will have the ability to request it from the server.
			\item \textbf{Server}
			\\A server will be running at all times that listens for incoming connections, if one tries to connect to the server a request object is sent to the server along with all the login details, or with a request to retrieve a lost or forgotten password, or to create a new account. If the server accepts the request it returns either with successful login moving to the menu screen or the password.
			\item \textbf{Menu Screen}
			\\There will be a main menu from which the settings for the game, such as difficulty, video quality, audio settings and game difficulty can be altered. Once all selections and settings have been made, the object will proceed to call the load game and request will be sent and if all is correct and the validity of the settings are checked, the game object will return with success and proceed to the next state, or an exit request is sent and the game is closed.
			\item \textbf{Settins Screen}
			\\This object will receive a request from the menu screen and respond with success and return the settings object which will allow the changing of settings and sending through the changed settings to the game object to update the state. Once the game object has responded with a successful update the object can either send a return request to the menu screen or to wait for more changes.
			\item \textbf{Game object}
			\\This will be the main object which will send requests and updates to the server database each time an event takes place, such as a kill is made or experience is gained, the server will receive an object with a value to update corresponding to the object's id as each object created will have thier own id linked to that of player who signed in. A request can be sent to the menu screen, if successful the menu screen will appear, if the game object crashes due to power failure or server connection issues, the state will be saved and when the object is reinstantiated, will resume in the dead state and will carry have to start from the safe zone.
			\end{enumerate}
		
		
		
		\vspace{0.2in}
		
		
		\section*{\colorbox{blue}{\makebox[\textwidth-2\fboxsep][l]{\bfseries\color{white} Architectural constraints}}} \addcontentsline{toc}{section}{Architectural constraints}
		\vspace{0.1in}
		
			\subsection*{Technologies}
			\addcontentsline{toc}{subsection}{Technologies}
			\vspace{0.1in}
			
				\begin{enumerate}					
					\item \textbf{Graphics}
					\\OpenGL 3.0 - 4.4 will be used as the rendering language within the game engine.
					
					\item \textbf{Audio}
					\\OpenAL will be used for audio within the game.
					
					\item \textbf{Development language}
					\\The desktop \& mobile application as well as the server will be implemented using JavaSE.
					
					\item \textbf{LDAP integration}
					\\Apache oltu is a Java implemented OAuth library that will be the used to communicate with external LDAP servers.
					
					\item \textbf{Networking}
					\\jSpiderMonkey will be used to handle client side networking within the jMonkeyEngine framework.
					
					\item \textbf{Persistence}
					\\MySQL will be used to store all data which is accessible through a JDBC connection.
				\end{enumerate}
				
			\subsection*{Architectural patterns/frameworks}
			\addcontentsline{toc}{subsection}{Architectural patterns/frameworks}
			\vspace{0.1in}
			
				\begin{enumerate}
					\item \textbf{Game Engine}
					\\jMonkeyEngine 3.0 is a Java Framework that will be used to implement the game.
					
					\item \textbf{Client Server}
					\\A client server pattern should be used between the external application and the backend server.
					
					\item \textbf{Layering}
					\\The backend server should implement layering from connections to clients down to the LDAP and MySQL server connections.
				\end{enumerate}
			
		\vspace{0.2in}
		
\end{document}