\documentclass[letterpaper]{article}
\usepackage{amsmath}
\usepackage{tikz}
\usepackage{epigraph}
\usepackage{lipsum}
\usepackage{hyperref}
\usepackage{tocloft}
\usepackage{graphicx}
\usepackage{float}

\usepackage{setspace, amsmath}

\usepackage[centering,includeheadfoot,margin=2cm]{geometry}
\usepackage{xcolor}
\usepackage{calc,blindtext}

\renewcommand\epigraphflush{flushright}
\renewcommand\epigraphsize{\normalsize}
\setlength\epigraphwidth{0.6\textwidth}

\definecolor{titlepagecolor}{cmyk}{1,.60,0,.40}

\DeclareFixedFont{\titlefont}{T1}{ppl}{b}{it}{1.0in}

\makeatletter
\def\printauthor{%
    {\large \@author}}
\makeatother
\author{%
    Nico Taljaard \\
    10153285 \vspace{20pt} \\
    Gerhard Smit \\
    12282945 \vspace{20pt} \\
    Martin Schoeman \\
    10651994 \\
}

% The following code is borrowed from: http://tex.stackexchange.com/a/86310/10898

\newcommand\titlepagedecoration{%
	\begin{tikzpicture}[remember picture,overlay,shorten >= -10pt]
	
		\coordinate (aux1) at ([yshift=-15pt]current page.north east);
		\coordinate (aux2) at ([yshift=-410pt]current page.north east);
		\coordinate (aux3) at ([xshift=-4.5cm]current page.north east);
		\coordinate (aux4) at ([yshift=-150pt]current page.north east);
		
		\begin{scope}[titlepagecolor!40,line width=12pt,rounded corners=12pt]
			\draw
			  (aux1) -- coordinate (a)
			  ++(225:5) --
			  ++(-45:5.1) coordinate (b);
			\draw[shorten <= -10pt]
			  (aux3) --
			  (a) --
			  (aux1);
			\draw[opacity=0.6,titlepagecolor,shorten <= -10pt]
			  (b) --
			  ++(225:2.2) --
			  ++(-45:2.2);
		\end{scope}
			\draw[titlepagecolor,line width=8pt,rounded corners=8pt,shorten <= -10pt]
			  (aux4) --
			  ++(225:0.8) --
			  ++(-45:0.8);
		\begin{scope}[titlepagecolor!70,line width=6pt,rounded corners=8pt]
			\draw[shorten <= -10pt]
			  (aux2) --
			  ++(225:3) coordinate[pos=0.45] (c) --
			  ++(-45:3.1);
			\draw
			  (aux2) --
			  (c) --
			  ++(135:2.5) --
			  ++(45:2.5) --
			  ++(-45:2.5) coordinate[pos=0.3] (d);   
			\draw 
			  (d) -- +(45:1);
		\end{scope}
	\end{tikzpicture}
}

\begin{document}

\begin{titlepage}

\noindent
\titlefont Laminin \par
\epigraph{ XGame - Derivco \\ Corspe Slasher \\ Software Architecture Specification.}%
{\textit{ 01/08/2014 }\\ \textsc{ }}
\null\vfill
\vspace*{4cm}
\noindent
\hfill
\begin{minipage}{0.35\linewidth}
    \begin{flushright}
        \printauthor
    \end{flushright}
\end{minipage}
%
\begin{minipage}{0.02\linewidth}
    \rule{1pt}{125pt}
\end{minipage}
\titlepagedecoration
\end{titlepage}

% % % % % % % % % % % % % % %
% 							%
%	Remainder of document	%
% 							%
% % % % % % % % % % % % % % % 

	\newpage
		{\LARGE \bf Change Log}\\[2em]
		
		\begin{tabbing}
			\hspace*{2.5cm}\=\hspace*{2.5cm}\=\hspace*{8cm}\=\hspace*{3cm} \kill
			28/07/2014	\> Version 1.0	\> Document Created 							\> Nico Taljaard \\
			29/07/2014	\> Version 1.0	\> Added game class diagram						\> Nico Taljaard \\
			29/07/2014	\> Version 1.0	\> Added game use case diagram					\> Nico Taljaard \\
		\end{tabbing}
		
	\newpage
		\renewcommand\contentsname{TABLE OF CONTENTS}
		\newcommand\contentsnameLC{\colorbox{blue}{\makebox[\textwidth-2\fboxsep][l]{\bfseries\color{white} Table of Contents}}}
		
		\renewcommand{\cftdot}{}
		\hypersetup{linktocpage}
		\tableofcontents
		
		\begin{flushleft}
			\LARGE\href{https://github.com/njTaljaard/Laminin_CorpseSlasher/}{Git repository: Laminin - Corpse Slasher}
		\end{flushleft}
		
	\newpage
		
		\section*{\colorbox{blue}{\makebox[\textwidth-2\fboxsep][l]{\bfseries\color{white} Functional Requirements }}} \addcontentsline{toc}{section}{Functional Requirements}
		\vspace{0.1in}
		
			\subsection*{Introduction:}
			\addcontentsline{toc}{subsection}{Introduction}
			\vspace{0.1in}
			
			
			
			\vspace{0.2in}
			\subsection*{Required Functionality:}
			\addcontentsline{toc}{subsection}{Required Functionality}
			\vspace{0.1in}
			
			
			
			\vspace{0.2in}
			\subsection*{Use Case prioritization / services contracts:}
			\addcontentsline{toc}{subsection}{Use Case prioritization / services contracts}
			\vspace{0.1in}
			
				\vspace{0.2in}
				\subsubsection*{Game Diagrams:}
				\addcontentsline{toc}{subsubsection}{Game Diagrams}
				\vspace{0.2in}
				
					\begin{figure}[H]
					\centering
					\includegraphics[width=140mm]{UML_Diagram/Use_Case/Game_Process.jpg}
					\caption{Update process between frames}
					\end{figure}
					
					\begin{itemize}
					\item Pre-Conditions: \\
						Game is still running. \\
						Update time is provided. \\
						Player character is accessible. \\
						Mob characters are accessible.
					\item Post-Conditions: \\
						Game scene has updated. \\
						Character action updated. \\
						Mobs actions updated.
					\end{itemize}
					
					\vspace{0.2in}
					\begin{figure}[H]
					\centering
					\includegraphics[width=140mm]{UML_Diagram/Use_Case/Player_Actions.jpg}
					\caption{Actions available to player}
					\end{figure}
					
					\begin{itemize}
						\item Pre-Condition: \\
							Character handler is accessible. \\
							Character repositioning direction available. \\
							Player animations channel is accessible. \\
							Player attack registered.
						\item Post-Condition: \\
							Character position has been updated. \\
							Player animation has be set successfully. \\
							Player attack action has occurred and processed.
					\end{itemize}
					
					\vspace{0.2in}
					\begin{figure}[H]
					\centering
					\includegraphics[width=140mm]{UML_Diagram/Use_Case/Mob_Actions.jpg}
					\caption{Actions available to mobs}
					\end{figure}
					
					\begin{itemize}
						\item Pre-Condition: \\
							Character handler is accessible. \\
							Character repositioning direction available. \\
							Mob aggro detection available. \\
							Mob aggro obtained flagged. \\
							Mob aggro loss flagged. \\
							Mob animation channel is accessible. \\
							Mob attack registered.
						\item Post-Condition: \\
							Character position updated. \\
							Mob aggro process and state updated. \\
							Mob animation has be set successfully. \\
							Mob attack action has occurred and processed.
					\end{itemize}
					
					\vspace{0.2in}
					\begin{figure}[H]
					\centering
					\includegraphics[width=140mm, height=100mm]{UML_Diagram/Use_Case/Update_Scene.jpg}
					\caption{Update of scene elements}
					\end{figure}
					
					\begin{itemize}
						\item Pre-Condition: \\
							Sky control available. \\
							Update value provided.
						\item Post-Condition: \\
							Sun \& Stars updated. \\
							Sun direction updated. \\
							Sun on the water updated. \\
							Time of day updated. 
							
					\end{itemize}
					
				\vspace{0.2in}
				\subsubsection*{User-interface Diagrams:}
				\addcontentsline{toc}{subsubsection}{User-interface Diagrams}
				\vspace{0.2in}
				
				
				
				\vspace{0.2in}
				\subsubsection*{Server Diagrams:}
				\addcontentsline{toc}{subsubsection}{Server Diagrams}
				\vspace{0.2in}
					
			\vspace{0.2in}
			\subsection*{Process specifications:}
			\addcontentsline{toc}{subsection}{Process specifications}
			\vspace{0.1in}
			
			
			
			\vspace{0.2in}
			\subsection*{Domain Objects:}
			\addcontentsline{toc}{subsection}{Domain Objects}
			\vspace{0.1in}
			
				\vspace{0.2in}
				\subsubsection*{Class Diagrams:}
				\addcontentsline{toc}{subsubsection}{Class Diagrams}
				\vspace{0.1in}
				
					\begin{figure}[H]
					\centering
					\includegraphics[width=180mm]{UML_Diagram/Class/Game_Classes.jpg}
					\caption{Game level class diagram}
					\label{overflow}
					\end{figure}
					
				\vspace{0.2in}
				\subsubsection*{State Diagrams:}
				\addcontentsline{toc}{subsubsection}{State Diagrams}
				\vspace{0.1in}
					
					
					
				\vspace{0.2in}
				\subsubsection*{Activity Diagrams:}
				\addcontentsline{toc}{subsubsection}{Activity Diagrams}
				\vspace{0.1in}
					
					
					
				\vspace{0.2in}
				\subsubsection*{ERD Diagrams:}
				\addcontentsline{toc}{subsubsection}{ERD Diagrams}
				\vspace{0.1in}
					
					
		\vspace{0.2in}
		
		\section*{\colorbox{blue}{\makebox[\textwidth-2\fboxsep][l]{\bfseries\color{white} Application Design }}} \addcontentsline{toc}{section}{Application Design}
		\vspace{0.1in}
		
		
		
		\vspace{0.2in}
		
		\section*{\colorbox{blue}{\makebox[\textwidth-2\fboxsep][l]{\bfseries\color{white} Code Documentation }}} \addcontentsline{toc}{section}{Code Documentation}
		\vspace{0.1in}
		
			\subsection*{Game documentation:}
			\addcontentsline{toc}{subsection}{Game documentation}
			\vspace{0.1in}
			
			\vspace{0.2in}
			\subsection*{User-interface documentation:}
			\addcontentsline{toc}{subsection}{User-interface documentation}
			\vspace{0.1in}
			
			\vspace{0.2in}
			\subsection*{Server documentation:}
			\addcontentsline{toc}{subsection}{Server documentation}
			\vspace{0.1in}
		
		\vspace{0.2in}
		
\end{document}